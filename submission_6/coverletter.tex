\documentclass{letter}
\usepackage[margin=1.5in]{geometry}
\usepackage{hyperref} 
\signature{Anna Trendl}
\address{Department of Psychology\\ University of Warwick \\ Gibbet Hill Road, Coventry \\ CV4 7AL, UK \\ \href{mailto:a.trendl@warwick.ac.uk}{a.trendl@warwick.ac.uk}}
\begin{document}

\begin{letter}
{}

%The Lancet Public Health is a monthly, open access, online-only journal working in partnership with our existing family of Lancet titles. Our focus is to attract, review, and publish quickly high-quality original research that contributes to advancing public health practice and policy making worldwide. The journal publishes Research Articles, Editorials, Comments, and Correspondence.
%
%The Lancet Public Health is committed to tackling the most pressing issues across all aspects of public health. In particular, and in line with the values and vision of the weekly Lancet journal, we have a strong commitment to using science as an important means to improve health equity and social justice. We take a broad and inclusive approach to public health, and we especially wish to emphasise our interest in interdisciplinary research.


\opening{Dear Audrey Ceschia,} % Addressed to the editor in chief

Please consider our manuscript ``The link between national football tournaments and alcohol-related domestic abuse - evidence from England'' for publication as a Research Article in \textit{The Lancet Public Health}. Domestic abuse is a growing public health policy concern around the world and more research is needed to understand the contexts which give rise to this criminal behaviour. Previous studies have suggested a potential link between football and domestic abuse, but the nature of this relationship, and the extent to which alcohol contributes to it has not yet been investigated. Exploring the pathways through which football tournaments increase propensity for violence will deepen our understanding of the environments that precipitate domestic abuse, and can inform the design of effective policy measures.


Our results show that the number of reported alcohol-related domestic abuse cases increases by 61\% following an England victory in a national football tournament, while there is no comparable effect on non-alcohol related domestic abuse. This effect is specific to male-to-female cases, and its time course is strongly consistent with a causal link between England's victory and an increase in alcohol-related domestic abuse. We also find a comparable increase in other, violent, male to female, alcohol-related offences on England win days. These results strongly implicate masculinity construction and alcohol consumption as the pathway through which football increases male-perpetrated violence against women. The effect survives various robustness checks, including a re-analysis of data used in a previous study. 

Our study significantly advances our understanding of the link between the world's most popular sport and domestic abuse, has important implications for public health policy, and we believe it fits the journal's interdisciplinary profile. We confirm that this manuscript has not been published elsewhere, and is not under consideration by another journal.

Thank you for your time and consideration.

\closing{Sincerely,}



\end{letter}
\end{document}
