\documentclass{letter}
\usepackage[margin=1.5in]{geometry}
\usepackage{hyperref} 
\signature{Anna Trendl}
\address{Department of Psychology\\ University of Warwick \\ Gibbet Hill Road, Coventry \\ CV4 7AL, UK \\ \href{mailto:a.trendl@warwick.ac.uk}{a.trendl@warwick.ac.uk}}
\begin{document}

\begin{letter}
{}

%All submissions should be accompanied by a cover letter that includes the following key points for the Editors:
%
%    Statement on why the work is appropriate for Palgrave Communications;
%    Clear indication of the research question/knowledge gap the work sets out to address;
%    Outline of the paper’s key aims and key conclusions;
%    Short description of how the paper builds on, and advances, existing thinking in the relevant field(s) of scholarship.
%
%Authors should also indicate whether they have had any prior discussions with a Palgrave Communications Editor or Editorial Board Member about the work described in the manuscript.

%Authors should provide a cover letter that includes the affiliation and contact information for the corresponding author. Authors should briefly explain why the work is considered appropriate for Palgrave Communications. Authors are asked to suggest the names and contact information for reviewers and they may request the exclusion of certain referees. Please ensure that your cover letter also includes suggestions for Editorial Board members who would also be able to review or advise on your submission. Finally, authors should indicate whether they have had any prior discussions with a Palgrave Communications Editorial Board Member about the work described in the manuscript.


%Actively welcomed for submission is research on agenda-setting issues, grand societal challenges and emerging areas of thinking, irrespective of the field of study. This also includes research that reflects on, or seeks to inform, policymaking of all types. 
%
%Palgrave Communications additionally welcomes interdisciplinary research that makes an explicit and valuable contribution to the advancement of the humanities and/or social sciences. This includes research arising in, or informed by, the physical, life, clinical and environmental sciences in areas such as the medical humanities, digital humanities, environmental sociology, and complex network analysis. 

\opening{Dear Editors,} % Addressed to the editor in chief

Please consider our manuscript ``The role of alcohol in the link between football and domestic abuse: evidence from England'' for publication as an Article in \textit{Palgrave Communications}. Previous studies have suggested a link between football and domestic abuse in the context of England, but the exact nature of this relationship, including the extent to which alcohol alters it, has not yet been explored. Understanding the characteristics of the link between football and domestic abuse will deepen our understanding of the pathway through which football increases propensity for violence, and can inform preventive policy measures.


Our results show that the number of reported alcohol-related domestic abuse cases increases by 61\% following an England victory
in a national football tournament, while there is no effect on non-alcohol related domestic abuse. This effect is specific to male-to-female cases, and its time course is strongly consistent with a causal link between England's victory and an increase in alcohol-related domestic abuse. We also find a comparable increase in other, violent, male to female, alcohol-related offences on England win days. These results strongly implicate masculinity construction and alcohol consumption as the pathway through which football increases male-perpetrated violence against women. In addition, we find that national rugby tournaments have no comparable effects on the reported number of domestic abuse cases. To test the robustness of the win-effect, our study also involves a re-analysis of data used in a previous study. 

We would like to suggest Kath Woodward as a potential reviewer. We confirm that this manuscript has not been published elsewhere, is not under consideration by another journal, and had not been discussed with a Palgrave Communications Editorial Board Member.

Thank you for your time and consideration.

\closing{Sincerely,}



\end{letter}
\end{document}
